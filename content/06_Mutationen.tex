\section{Mutationen}\label{sec:mutation}
Um neue Lösungen zu finden, werden die Eltern nicht nur kombiniert sondern auch zu
einer gewissen Wahrscheinlichkeit mutiert. Doch nicht jede Mutation ist für das Konzertproblem 
sinnvoll. 
\subsection{Analyse und Beschreibung verschiedener Mutationen}
\paragraph{Flip}
Bei dieser Mutation wird der Wert eines Bits verändert. Auf dieses Problem bezogen, 
bedeutet dies, dass ein Stück in die Setliste mit aufgenommen oder aus 
der Setliste gestrichen wird. 
\begin{enumerate}
    \item Entfernen eines Stückes: 
    \begin{itemize}
        \item Die Dauer des Auftritts wird kürzer, dies hat sehr wahrscheinlich eine Bestrafung und 
            damit eine Reduzierung der Fitness zur Folge. 
        \item Die Schwierigkeiten $s, s_{hoch}$, $ s_{tief}$ und $s_{holz}$ sinken ebenso. Dies führt allerdings 
            zu einer Steigerung der Fitness. 
    \end{itemize}
    \item Hinzufügen eines Stückes:
    \begin{itemize}
        \item Die Dauer des Auftritts wird länger, auch dies hat eine Reduzierung der Fitness zu Folge, falls die Dauer bereits das Maximum erreicht hat.
        \item Die Schwierigkeiten $s, s_{hoch}$, $ s_{tief}$ und $s_{holz}$ steigen ebenso, was auch zur Reduzierung 
            der Fitness führt. 
    \end{itemize}
\end{enumerate}
Dies bedeutet, dass die Fitness nach einem Flip nur steigen kann, wenn ein kurzes aber schwieriges Stück aus 
der Setliste entfernt wird. Allerdings könnte eine zunächst schlechtere Fitness den Algorithmus aus 
einem lokalen Optimum befreien.

\paragraph{Swap}
Beim Swap werden die Positionen von zwei Bits vertauscht. 
Daraus ergeben sich folgende Möglichkeiten für das Konzertproblem. 
\begin{enumerate}
    \item Fall \{$1 \longleftrightarrow 1$\}: Beide Stücke verbleiben in der Setliste.
    \item Fall \{$0 \longleftrightarrow 1$\} : Ein Stück wird von der Setliste gestrichen, dafür wird ein anderes Stück aufgenommen.
    \item Fall \{$0 \longleftrightarrow 0$\} : Beide Stücke sind nicht auf der Setliste.
\end{enumerate}

Fall 1 und Fall 3 sorgen weder für eine Verschlechterung noch zu einer Verbesserung. Dennoch sollten diese 
Fälle vermieden werden, da keine echte Mutation stattfindet. \\
Denkbar wäre es die zufällige Wahl der beiden Bits zu manipulieren. Das erste Bit wird aus der Menge der 
von der Setliste ausgeschlossenen Stücke gewählt und das zweite Bit aus der Menge der Setliste. 
Somit würde gewährleistet werden, dass bei jeder Swap-Mutation eine echte Mutation stattfindet. 
Der Austausch eines Stückes auf der Setliste mit einem anderen, ist für das Konzertproblem sehr sinnvoll, 
da die Dauer nur um die Differenz der Stückdauer verändert wird und nicht wie im Fall eines Flips um die gesamte
Länge eines Stückes. \\
Der Algorithmus hat durch den Swap die Möglichkeit die Population zu verbessern oder zu verschlechtern. 
Beide Fälle können sich positiv auf die Fitness des Endergebnisses auswirken. 

\paragraph{Shuffel}
Beim Shuffel werden alle Werte neu vermischt, daher sorgt auch der Shuffel für eine gleichbleibende Anzahl an Stücken.
Der Unterschied zum \emph{Swap} ist der Umfang. Es wird nicht nur ein Stück ausgetauscht, sondern sehr viele. 
Das bedeutet, dass diese Mutation eine starke Änderung des Genotyps zur Folge hat. \\

Alle drei Mutationen haben ihre Berechtigung zur Lösung des Konzertproblems. Allerdings sollte 
der Flip nicht so oft sattfinden wie der Swap, da dieser die Fitness positiver beeinflusst. 
Dies kann mit Hilfe von Wahrscheinlichkeiten erreicht werden. Eine Zufallszahl kann die Wahrscheinlichkeit 
für das Stattfinden einer Mutation bestimmen. 
Als initiale Einstellung könnten die Wahrscheinlichkeiten wie folgt gewählt werden: 
\begin{enumerate}
    \item $50\%$: Swap
    \item $25\%$: keine Mutation
    \item $20\%$: Shuffel
    \item $5\%$: Flip
\end{enumerate}
Diese Rangordnung sorgt dafür, dass die wahrscheinlichste Mutation ein Swap ist und ein Flip eher selten ausgeführt wird.
