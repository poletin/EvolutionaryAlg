\section{Bewertung}\label{sec:fitness}
Die Fitnessfunktion soll die so auswählen, dass:
\begin{itemize}
    \item Die Dauer $d$ der Stücke möglichst genau der Auftrittszeit 120 Minuten entspricht.
    \item Die allgemeine Schwierigkeit $s$ möglichst minimiert wird. Hierbei ist zu erwähnen, dass die hier
        angegebene Schwierigkeit nicht zwingend der Durchschnitt der Registerschwierigkeiten darstellt, denn
        beispielsweise im Bezug auf das Zusammenspiel kann die allgemeine Schwierigkeit höher sein.
    \item Die Stimmung $m$ und die Bekanntheit $n$ möglichst maximiert wird.
    \item Die Schwierigkeit für Hochblech $s_{hoch}$, Tiefblech $s_{tief}$ und Holzblasinstrumente $s_{holz}$ möglichst ausgeglichen ist.
\end{itemize}
Um die Fitness des Phenotyps zu brechnen, müssen zunächst einige Basiswerte berechnet werden.
Für die allgemeine Schwierigkeit sowie für die Schwierigkeiten der einzelnen Register, zeigt dies \autoref{eqn:basis}. Hierbei werden alle Werte des
Phenotyps addiert. Durch den Bruch werden die Werte zusätzlich auf einen Wert zwischen $0$ une $1$ normiert.
Die Formeln für $m(x)$ und $n(x)$ sind identisch.
\begin{equation}
    s(x) = \frac{\sum_{i=0}^{n} x_i * (10 - s(i)) }{n * 10}
    \label{eqn:basis}
\end{equation}

Die Dauer sollte möglichst der geforderten Auftrittszeit entsprechen. Wird die Auftrittszeit nicht erfüllt
oder überschritten, soll sich dies negativ auf die Fitness des Phenotyps auswirken. Aus diesem Grund wird das
Ergebnis in \autoref{eqn:penalty} negiert. Auch in dieser Formel werden zunächst alle Differenzen zur geforderten
Spielzeit addiert. Die Bildung des Quadrats sorgt dafür, dass sowohl unterschreiten wie auch überschreiten
positive Werte ergibt.
Auch in \autoref{eqn:penalty} sorgt der Bruch für eine Normierung der Werte.

\begin{equation}
    penalty(x) = -\left( \frac{\sum_{i=0}^{n}x_i * (d(i)-120)}{120}\right)^2
    \label{eqn:penalty}
\end{equation}

Um einen Wert für die faire Verteilung der Schwierigkeit zu berechnen ist es zunächst notwendig, den Durchschnitt der
Schwierigkeit der Register zu berechnen. Dies zeigt \autoref{enq:avg}

\begin{equation}
    avg(x) = \frac{s(x)_{hoch} + s(x)_{tief} + s(x)_{holz}}{3}
    \label{enq:avg}
\end{equation}

Anschließend kann dieser Wert verwendet werden, um, wie in \autoref{eqn:fair} dargestellt,
den mittleren quadratischen Fehler zu berechnen.
\begin{equation}
    g(x) = \frac{((avg(x) - s(x)_{hoch}) + (avg(x) - s(x)_{tief}) +(avg(x) - s(x)_{holz}))^2}{3}
    \label{eqn:fair}
\end{equation}

Durch die Vorberechnung der einzelnen Werte ist die eigentliche Fitnessfunktion stark reduziert.
\autoref{eqn:fitness} zeigt die Fitnessfunktion mit allen zuvor beschriebenen Kriterien.
\begin{equation}
    f(x) = w_p * p(x) + w_s * s(x) + w_m * m(x) + w_n * n(x) +  w_g * g(x)
    \label{eqn:fitness}
\end{equation}

% \lstinputlisting[
%     float,
%     floatplacement=H,
%     caption=Implementierung der Fitnessfunktion,
%     label=lst:fitness
% ]{fitness.java}