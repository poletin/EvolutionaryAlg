\section{Fazit und Ausblick}\label{sec:fazit}
\paragraph{Fazit}
Dieser Programmentwurf ist das Konzept für das Konzertproblem, welches 
eine Setliste für einen Auftritt erstellt und dabei die Dauer und die Schwierigkeit, 
aber auch die Stimmung und den Bekanntheitsgrad der Stücke beachtet. 
Die Konzepte und Strategien müssen hierbei jedoch noch praktisch getestet und 
gegebenenfalls angepasst werden. \\
\paragraph{Ausblick}
Aufbauend auf diesem Konzept wäre es denkbar mit Hilfe einer Permutations-Repräsentation 
auch die Reihenfolge der Stücke zu optimieren. Dies wäre dem Traveling Salseman Problem
ähnlich. Jedoch müsste hierbei nicht die Strecke der Orte minimiert werden, sondern es müssten 
viele andere Faktoren berücksichtigen. Ein Auftritt ist immer in einer ähnlichen Form 
aufgebaut:
\begin{enumerate}
    \item Das Einleitungsstück sollte nicht zu schwierig sein, aber dennoch die Aufmerksamkeit 
        des Publikums fangen.
    \item möglichst unterschiedliche Stücke, das heißt nach einigen schnellen Stücken vielleicht ein 
        langsames etc. 
    \item Gegen Ende des Auftritt einige Stimmungslieder bzw. Lieder mit hohem Bekanntheitsgrad
    \item Eine stimmungsvolle Zugabe und anschließend eine ruhige Zugabe. 
    \item Zum Abschluss muss, zumindest bei den Auftritten in Baden-Württemberg, das Badnerlied gespielt werden. 
\end{enumerate}

Dieser Grobablauf, könnte als Schablone für das Optimierungsproblem dienen. Die Stücke erhalten Eigenschaften, 
wie zum Beispiel \textit{mögliches Einleitungsstück} anhand derer über die Fitness entschieden wird. 

