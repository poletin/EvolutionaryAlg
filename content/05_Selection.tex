\section{Kombination}\label{sec:eltern}
Da der Genstring in diesem Konzertproblem sehr lange ist, 73 Elemente, bietet es sich an einen Multi-Point-Crossover zu verwenden.
Dabei werden die beiden Eltern an $n$ zufällig gewählten Stellen getrennt und auf zwei neue Genstrings aufgeteilt.
\autoref{img:multiCrossover} zeigt anschaulich wie ein 3-Point-Crossover funktioniert. An je mehr Punkten
der Genstring geteilt und wieder zusammengesetzt wird desto einen größer wird der Unterschied zwischen Eltern und Kindern.
Das Crossover an 3 Punkten bietet sich in dieser Anwendung an, da so der Unterschied nicht zu groß aber auch nicht zu klein ist.

\begin{figure}[h]
    \begin{minipage}{\textwidth}
	    \centering
        \includegraphics{multiPointCrossover.PNG}
	    \caption{Multi-Point-Crossover {http://www.geatbx.com}}
        \label{img:multiCrossover}
    \end{minipage}
\end{figure}

\section{Selektion}
Für die Selektion der Eltern bietet sich die \emph{Tournament Selection} an.
Diese wählt zufällig mehrere Chromosomen aus und selektiert das Chromosom mit der besten Fitness. Das Verfahren der
Tournament Selection ist in \autoref{img:tournament} dargestellt.


\begin{figure}
    \begin{minipage}{\textwidth}
        \includegraphics[width=\textwidth]{tournament_selection.jpg}
        \caption{Tournament Selection {https://www.tutorialspoint.com}}
        \label{img:tournament}
    \end{minipage}
\end{figure}

Da eine Population etwa 25 Gene besitzt, sollten für die Tournament Selection nicht mehr als 4 Chromosomen zufällig gewählt werden.
Würden zu viele Chromosomen gewählt werden, könnten auch einfach die besten zwei Gene der Population ausgewählt werden.
Dies würde zu wenig Veränderung der Gene führen und könnte dafür sorgen, dass nicht das globale sondern
lediglich ein lokales Optimum gefunden wird.

Ein Ansatz der genauer untersucht werden sollte ist es, an einem Zeitpunkt gegen Ende der
Evolution die Tournament Selection mit einer Zufall-Selektion auszutauschen.
Dies hat den Grund, dass zu diesem Zeitpunkt angenommen werden kann, dass alle
Gene eine ähnliche Fitness besitzen. Diese Idee könnte zu einer Optimierung der Selektion führen, dies 
müsste allerdings in einem Prototypen analysiert werden. 