\section{Einleitung}\label{sec:einleitung}
In einem kleinen Ensemble mit regelmäßigen Auftritten, kommt es immer wieder zu Diskussionen 
bezüglich der Setliste. Neben der Länge des Musikstückes sind bei der Auswahl der Stücke einige 
weitere Eigenschaften zu beachten. Eines ist beispielsweise die Anstrengung für das Hochblech. 
Besonders bei langen Auftritten, ist es wichtig nicht zu viele Stücke auszuwählen, die 
für einzelne Register besonders anstrengend sind. Die Anstrengung soll für alle Register ausgeglichen sein.
\subsection{Beschreibung des Problems}
 Dieser Programmentwurf, stellt die Basis für ein System dar, dass anhand verschiedener Kriterien eine 
 Auswahl von Stücken zusammenstellt, die möglichst alle Kriterien optimiert und dabei die 
 geforderte Auftrittszeit nicht unterschreitet. Diese Problemstellung lehnt somit an das \textit{Knappsack-Problem} an.
\subsection{Kurze Beschreibung des Ensembles}
Das Ensemble besteht aus: 
\begin{itemize}
    \item 2 Trompeten (Hochblech)
    \item 2 Saxophone (Holzbläser)
    \item 1 Posaune (Tiefblech)
    \item 1 Tuba (Tiefblech)
    \item 1 Schlagzeug 
\end{itemize} 

Für einen Auftritt stehen ** Stücke zur Verfügung die zusammengestellt werden können. Dabei ist zu erwähnen, dass keines
der Stücke bei einem Auftritt nicht gespielt werden kann, da die Stücke bereits für die Auftritte angepasst sind.
Die Auftritte haben eine Dauer von 120 Minuten.