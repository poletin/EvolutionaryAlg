\section{Genotyp und Phenotyp}\label{sec:genotypPhenotyp}
\paragraph{Genotyp}
Der Genotyp ist ein Bitstring, dieser beschreibt, welche Lieder gespielt werden und welche nicht.
Dabei steht die 1 dafür, dass ein Lied ausgewählt wurde. \\
Der Genotyp ist demnach wie folgt definiert: 
\begin{equation}
    \label{eqn:genotyp}
    \begin{split}
        x_i &\in \{ 0,1 \} \\
        x_i &= 1 \Rightarrow \text{Stück $i$ wird in die Setliste aufgenommen} \\
        x   &= x_1x_2...x_n \in \{0,1\}^n\text{ mit $n = $ Anzahl der gespielten Stücke}
    \end{split}
\end{equation}

\paragraph{Phenotyp}
Der Phenotyp ist die Setliste mit allen Liederobjekten und wird für die
Berechnung der Fitness verwendet.
Er besitzt neben der Stückbezeichnung alle weiteren in \autoref{sec:settings}
beschriebenen Eigenschaften. \autoref{lst:phenotyp} zeigt beispielhaft die Java-Implementierung der Klasse
eines Phenotyps.

\lstinputlisting[
    float,
    floatplacement=H,
    caption=Klasse für einen Phenotyp,
    label=lst:phenotyp
]{phenotyp.java}
