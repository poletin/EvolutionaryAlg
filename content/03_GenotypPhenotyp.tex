\section{Genotyp und Phenotyp}\label{sec:genotypPhenotyp}
\subsection{Genotyp}
Der Genotyp ist ein Bitstring, der Beschreibt welche Lieder gespielt werden und welche nicht. 
Dabei steht die 1 dafür, dass ein Lied ausgewählt wurde und besitzt so die Form
\{0,1,1,1,0,0,0,1,1,1,1,...\}.

\subsection{Phenotyp}
Der Phenotyp ist die Setliste mit allen Liederobjekten und wird für die 
Berechnung der Fitness verwendet. 
Er besitzt neben der Stückbezeichnung alle weiteren in \autoref{sec:settings} 
beschriebenen Eigenschaften. \autoref{lst:phenotyp} zeigt die Implementierung des Phenotyps.

\lstinputlisting[
    float,
    floatplacement=H,
    caption=Implementierung des Phenotyps,
    label=lst:phenotyp
]{phenotyp.java}

\subsection{Population}